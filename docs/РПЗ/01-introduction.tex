\section*{Введение}
	\addcontentsline{toc}{section}{Введение}
	
	В настоящее время процент использования информационных технологий в различных сферах растёт с каждым годом, так, согласно исследованиям \cite{Russia-in-numbers} за 2018 год он вырос на $3\%$ по сравнению с предыдущим, и такая тенденция сохраняется на протяжении нескольких лет.\\

	На сегодняшний день использование такого рода технологий позволяет любой организации без препятствий взаимодействовать с большими потоками информации. Быстрый доступ к базам данных обеспечивает оперативность обмена материалами и согласованность всей работы в целом. \\
	
	Однако внедрение подобного подхода в различные инфраструктуры происходит не равномерно, например, в Распоряжении Правительства РФ "Об утверждении Стратегии развития охотничьего хозяйства в Российской Федерации до 2030 года" \cite{doc_problems} рассматривается проблема использования консервативных и неэффективных методов работы, которые затрудняют организацию и контроль. \\
	
	Объектом разработки курсового проекта было выбрано <<Общество охотников>>, осуществляющее возможность подачи заявки и покупки путёвок. \\
	
	\textbf{Целью} работы - спроектировать и реализовать базу данных для <<Общества охотников>>, и разработать Web-приложение для взаимодействия с базой данных.\\
	Выделены следующие \textbf{задачи}:
	\begin{enumerate}
		\item[1)] провести анализ существующих решений;
		\item[2)] формализовать задание и определить необходимый функционал;
		\item[3)] осуществить обзор существующих решений;
		\item[4)] провести анализ существующих СУБД;
		\item[5)] спроектировать базу данных для хранения и структурирования данных;
		\item[6)] реализовать спроектированную базу данных с использованием выбранной СУБД;
		\item[7)] разработать соответствующее Web-приложение.
	\end{enumerate}
	  